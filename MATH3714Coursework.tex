% !TEX TS-program = pdflatex
% !TEX encoding = UTF-8 Unicode

% This is a simple template for a LaTeX document using the "article" class.
% See "book", "report", "letter" for other types of document.

\documentclass[11pt]{article} % use larger type; default would be 10pt
\usepackage[svgnames]{xcolor}
\usepackage{listings}
\lstset{language=R,
    basicstyle=\small\ttfamily,
    stringstyle=\color{DarkGreen},
    otherkeywords={0,1,2,3,4,5,6,7,8,9},
    morekeywords={TRUE,FALSE},
    deletekeywords={data,frame,length,as,character},
    keywordstyle=\color{blue},
    commentstyle=\color{DarkGreen},
}
\usepackage[backend=biber]{biblatex}
\usepackage[utf8]{inputenc} % set input encoding (not needed with XeLaTeX)
\usepackage{listings}
\usepackage{amsmath}
%%% Examples of Article customizations
% These packages are optional, depending whether you want the features they provide.
% See the LaTeX Companion or other references for full information.

%%% PAGE DIMENSIONS
\usepackage{geometry} % to change the page dimensions
\geometry{a4paper} % or letterpaper (US) or a5paper or....
% \geometry{margin=2in} % for example, change the margins to 2 inches all round
% \geometry{landscape} % set up the page for landscape
%   read geometry.pdf for detailed page layout information

\usepackage{graphicx} % support the \includegraphics command and options

% \usepackage[parfill]{parskip} % Activate to begin paragraphs with an empty line rather than an indent

%%% PACKAGES
\usepackage{booktabs} % for much better looking tables
\usepackage{array} % for better arrays (eg matrices) in maths
\usepackage{paralist} % very flexible & customisable lists (eg. enumerate/itemize, etc.)
\usepackage{verbatim} % adds environment for commenting out blocks of text & for better verbatim
\usepackage{subfig} % make it possible to include more than one captioned figure/table in a single float
% These packages are all incorporated in the memoir class to one degree or another...

%%% HEADERS & FOOTERS
\usepackage{fancyhdr} % This should be set AFTER setting up the page geometry
\pagestyle{fancy} % options: empty , plain , fancy
\renewcommand{\headrulewidth}{0pt} % customise the layout...
\lhead{}\chead{}\rhead{}
\lfoot{}\cfoot{\thepage}\rfoot{}

%%% SECTION TITLE APPEARANCE
\usepackage{sectsty}
\allsectionsfont{\sffamily\mdseries\upshape} % (See the fntguide.pdf for font help)
% (This matches ConTeXt defaults)

%%% ToC (table of contents) APPEARANCE
\usepackage[nottoc,notlof,notlot]{tocbibind} % Put the bibliography in the ToC
\usepackage[titles,subfigure]{tocloft} % Alter the style of the Table of Contents
\renewcommand{\cftsecfont}{\rmfamily\mdseries\upshape}
\renewcommand{\cftsecpagefont}{\rmfamily\mdseries\upshape} % No bold!

%%% END Article customizations

%%% The "real" document content comes below...

\title{MATH3714 Coursework}
\author{Viet Dao\\email: mm16vd@leeds.ac.uk}
%\date{} % Activate to display a given date or no date (if empty),
         % otherwise the current date is printed 
% File is created and written to disk by the above package

\begin{document}
\maketitle
\newpage
\tableofcontents
\newpage

\section{Introduction}
We have been given a data frame $\textbf{A}_{393x9}$ which is the table of different cars with mpg, cylinders, displacement, horsepower, weight, acceleration, year, origin and name for a given car. Our goals is to be make a model that is capable of predicting mpg from our data given. Now we plit up $\textbf{A}_{393x9}$ into $\textbf{Y}_{393x1}$ which contains only mpg and $\textbf{Y}_{393x8}$ which contains everything in $\textbf{A}_{393x9}$ apart from mpg. This sets up our responds and explainatory variable.

\section{Initial Data Analysis}
In this prelimatory stage we want to investigate outliers and possible missing data in our dataframe $\textbf{A}_{393x9}$. The summary of our data is a useful point to start from:
\begin{lstlisting}
> dat = read.table("http://www1.maths.leeds.ac.uk/~charles/math3714/Auto.csv",
 header = T)
> View(dat)
> summary(dat)
mpg 		cylinders	displacement	 horsepower
Min.   : 9.0	Min.   :3.000	Min.   : 68.0 	Min.   : 46.0
1st Qu.:17.0	1st Qu.:4.000	1st Qu.:105.0	1st Qu.: 75.0
Median :23.0	Median:4.000	Median :151.0	Median : 94.0
Mean   :23.5	Mean   :5.468	Mean   :194.1	Mean   :104.5
3rd Qu.:29.0	3rd Qu.:8.000	3rd Qu.:267.0	3rd Qu.:125.0
Max.   :46.6	Max.   :8.000	Max.   :455.0	Max.   :230.0

weight		acceleration	year		origin
Min.   :1613	Min.   : 8.00	Min.   :18.00	Min.   :1.000
1st Qu.:2226	1st Qu.:13.70	1st Qu.:73.00	1st Qu.:1.000
Median :2807	Median :15.50	Median :76.00	Median :1.000
Mean   :2978	Mean   :15.52	Mean   :75.83	Mean   :1.578
3rd Qu.:3613	3rd Qu.:17.00	3rd Qu.:79.00	3rd Qu.:2.000
Max.   :5140	Max.   :24.80	Max.   :82.00	Max.   :3.000

name
amc matador       :  5
ford pinto        :  5
toyota corolla    :  5
amc gremlin       :  4
amc hornet        :  4
chevrolet chevette:  4
(Other)           :366
\end{lstlisting}
There are several problem with the data:
\begin{itemize}
\item First there is a problem with a data in the year. the summary says the earliest car made was in 18, is it 1918 or 2018?. On further inspection using View(dat) command we can see the name of that car is 'vw golf estate S 1.4 TSI' clearly from 2018 rather than 1918. This need to be changed from 18 to 118. \\
\textbf{Solution}
\begin{lstlisting}
> dat$year[dat$name=='vw golf estate S 1.4 TSI'] = 118
> View(dat)
\end{lstlisting}
This fix the year releasing date for vw golf estate S 1.4 TSI.\\
Note: there should be a space between 'vw' and 'golf' instead of weird symbol.
\item The second problem is the name of the cars. This problem lies in the make of the car and the name of the cars are in the same string hence we are not able to 'encode' this properly i.e. amc hornet, amc gremlin are almost identical but if we were to fit these values under the model it would be treated as different. From here, the name can be plit into two more groups, which is make of the car and name of the car. From there the make of the car can be encoded, similar to the origin of the car.\\
\textbf{Solution}\\
The first thing to notice in the 'name' header is that the first word is the 'make' of the car and the rest is the 'model' of the car. Now take the first word of the string and add it to make while for name remove the first word of the string.
\begin{lstlisting}
dat = read.table("http://www1.maths.leeds.ac.uk/~charles/math3714/Auto.csv", header = T, stringsAsFactors = F)
#---Addressing 2nd problem
#In order to achieved this I need to add an extra tag into the 
#dataframe which is "stringAsFactors=F".
#adding a extra entry called make which stands for the maker of the car.
dat$make = dat$name

#changing the string into the first word of the sring.
#Then attaching the first word of the string to make table.
for(string in dat$make){
  substring = strsplit(string, " ")[[1]]
  maker = substring[1]
  print(maker)
  dat$make[dat$make==string]=maker
}

#changing the string into every word apart from the first word.
for(string in dat$name){
  substring = strsplit(string, " ")[[1]]
  print(paste(substring[-1], collapse=' ' ))
  dat$name[dat$name==string]=paste(substring[-1], collapse=' ' )
}
\end{lstlisting}
This should produce a new table with 'make' and 'name'.\\
NOTE: when importing the table the 'stringAsFactors=F' is a must else this wouldn't work.

\item A problem that arise from spliting the `name` column into `name` and `make` is the fact that the `make` is a catergorical data and this need to be encoded i.e. convert catergory into integers, similarly to the origin which is a catergorical data but represented by 1-3.\\
\textbf{Solution: R-Code}
\begin{lstlisting}
>table(dat$make)
amc	audi	bmw	buick	cadillac	capri	chevroelt
27	7	2 	17	2 		1 	1 
chevrolet	chevy	chrysler	datsun 
43		3 	6 		23 
dodge	fiat	ford	hi	honda	maxda	mazda
28 	8	48 	1	13	2	10	
mercedes mercedes-benz	mercury	nissan 
1	 2		11	1 
oldsmobile	opel 	peugeot	plymouth	pontiac	renault
10		4	8	31		16	3
saab	subaru	toyota	toyouta	triumph
4	4	25	1	1 
vokswagen	volkswagen 	volvo	vw 
1		15		6	7 
\end{lstlisting}
The table produce is in the form of `maker` of the cars and directly below is the number of occurances in the table.

\item Another problem lies in the fact that the data use several acronyms for the name make i.e. chevrolet and chevy, vw and volkswagen etc... This is a problem since it adds unwated complexity to our data. Therefore the data needs to be changed.\\
\textbf{Solution R-Code}:
\begin{lstlisting}
#---Problem 3
#Changing the make to a proper make.
for(string in dat$make){
  if(string=="chevroelt" | string=="chevy"){
    dat$make[dat$make==string]="chevrolet"
  }
  if(string=="maxda"){
    dat$make[dat$make==string]="mazda"
  }
  if(string=="mercedes-benz"){
    dat$make[dat$make==string]="mercedes"
  }
  if(string=="toyouta"){
    dat$make[dat$make==string]="toyota"
  }
  if(string=="vokswagen"|string=="vw"){
    dat$make[dat$make==string]="volkswagen"
  }
}
\end{lstlisting}

\item The name of the vehicle is also a problem. This is beacause the vehical name is very unique and dependent on the maker of that car i.e. `100ls' is dependent on audi since only `audi' make cars with those names. This also poses the problem of that the name is so unique that it can cause over fitting. The solution is not delete the name column and only include the brand as one of our explainatory variable.
\begin{lstlisting}
#---Problem 4
dat$name=NULL
\end{lstlisting}
\end{itemize}
\end{document}
